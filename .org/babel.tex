% Created 2022-04-28 Thu 20:28
% Intended LaTeX compiler: pdflatex
\documentclass[11pt]{article}
\usepackage[utf8]{inputenc}
\usepackage[T1]{fontenc}
\usepackage{graphicx}
\usepackage{longtable}
\usepackage{wrapfig}
\usepackage{rotating}
\usepackage[normalem]{ulem}
\usepackage{amsmath}
\usepackage{amssymb}
\usepackage{capt-of}
\usepackage{hyperref}
\author{pol}
\date{\today}
\title{}
\hypersetup{
 pdfauthor={pol},
 pdftitle={},
 pdfkeywords={},
 pdfsubject={},
 pdfcreator={Emacs 28.1 (Org mode 9.5.2)}, 
 pdflang={English}}
\begin{document}

\tableofcontents

\#+title Emacs!

\section{Info}
\label{sec:orgfbc6ad0}
\subsection{Babel header}
\label{sec:orga085b2a}
Apply the same configuration above to every block
Tangle means where to put this files code to
mkdirp create directory if needed
\subsection{Updates on emacs}
\label{sec:org852232c}
\begin{itemize}
\item M-x list-packages
\item Press U to mark for upgrade
\item Press x to prompt for upgrade
\end{itemize}

\subsection{Performance}
\label{sec:org4332028}
\begin{itemize}
\item :hook - Package will load the first time one of the hooks
\item :bind - Package will load when a keybinding is used
\item :commands - When a command is used
\item :mode - First time a file with a particular extension is used
\item :after - load this package after other specific packages are loaded
\item :defer - defer loading until startup if no other options are used
see also:
with-eval-after-load 'package - execute body after package is loaded
\end{itemize}
\subsection{Mac-OS}
\label{sec:org30fc153}
Install with:
\begin{verbatim}
brew tap d12frosted/emacs-plus
brew install emacs-plus
\end{verbatim}

And remember to symlink, check how with
\begin{verbatim}
brew info emacs-plus
\end{verbatim}
\subsection{Guide to keybindings}
\label{sec:org17067e4}
\begin{itemize}
\item C-n C-p : in normal mode to go back and forth the clipboard
\item C-x o : emacs go between windows
\item C-h v : help on variables
\item C-h f : help on functions
\end{itemize}

\subsubsection{Prefixes}
\label{sec:org3133685}
\begin{itemize}
\item C-x: emacs keybindings
\item C-h: help
\item M-x: execute function
\item C-c: user-defined keybindings
\item C-SPC: more user defined keybindings
\end{itemize}

\subsection{General information}
\label{sec:org57fe0cf}
Use bookmarks to open specific files.
To get help on variable of function put corsor over it and press C-h v or C-h f

\subsection{Undo in Emacs}
\label{sec:org846c622}

C-r WONT WORK, cause we are not using a redo we have a undo stack
undo is treated like a normal command, therefore you can undo the undo.
undo's will go to the stack, if a move like "k" is done then you can undo again, this makes it possible to undo every single edit.
in typicall (global-undo-tree-mode) (evil-set-undo-system 'undo-tree)
if you are doing several undos and miss the "correct spot" and do anything at all which is not an undo command, you are stuck: you broke the chain of undos \url{https://www.emacswiki.org/emacs/RedoMode}

\subsection{Solving errors}
\label{sec:org87696f0}
\subsubsection{Package errors}
\label{sec:org6c5c9c6}
If package give error try:type list-packages command to update packages.
\subsubsection{Org errors}
\label{sec:orgd57f7e2}
Sometimes have to type org-mode so that it reloades
\subsubsection{Newly added packages can't be installed}
\label{sec:org0cc0868}
Could be because other packages are outdated, proceed to update them.
\section{Code}
\label{sec:org7308e8f}
\subsection{Startup config}
\label{sec:orgd77571e}
\subsubsection{Optimization with garbage collector}
\label{sec:orga0a2122}
\begin{verbatim}
;; The default is 800 kilobytes.  Measured in bytes.
(defvar last-file-name-handler-alist file-name-handler-alist)
(setq gc-cons-threshold 402653184
    gc-cons-percentage 0.6
    file-name-handler-alist nil)
\end{verbatim}

\subsubsection{Debug packages and see startup time}
\label{sec:orgc24924c}
\begin{verbatim}
(setq use-package-verbose t) ;; debug to see which packages load, and maybe shouldn't, should be off
\end{verbatim}
\subsubsection{Customize System to different file}
\label{sec:org7e8e8d1}
\begin{verbatim}
(setq custom-file "~/.emacs.d/custom.el")
(load custom-file)
\end{verbatim}
\subsection{Basic UI Settings}
\label{sec:orgf76ed69}
\begin{verbatim}
    ;; You will most likely need to adjust this font size for your system!
    (defvar runemacs/default-font-size 110)

    (setq inhibit-startup-message t) ; Disable startup menu
    (scroll-bar-mode -1) ; Disable the scrollbar
    (tool-bar-mode -1)
    ;(tooltip-mode -1) disable tooltips ;; (text displayed when hovering over an element)
    (set-fringe-mode 10) ; Make some space
    (menu-bar-mode -1) ;; remove top bar
    (cond ((eq system-type 'windows-nt)
	;; Windows-specific code goes here.
	)
	  ((eq system-type 'darwin)
	      (setq ring-bell-function ;; subtle mode line flash
		  (lambda ()
		      (let ((orig-fg (face-foreground 'mode-line)))
		      (set-face-foreground 'mode-line "#F2804F")
		      (run-with-idle-timer 0.1 nil
					  (lambda (fg) (set-face-foreground 'mode-line fg))
					  orig-fg))))
	)
	((eq system-type 'gnu/linux)
	 (setq visible-bell t)
	))

    ;; (setq scroll-step            1
    ;;     scroll-conservatively  10000) ;; scroll line by line not like a fucking degenerate
    ;; (setq smooth-scroll-margin 4) ;; margin like in vim
;;
;;; Scrolling

(setq hscroll-margin 2
      hscroll-step 1
      ;; Emacs spends too much effort recentering the screen if you scroll the
      ;; cursor more than N lines past window edges (where N is the settings of
      ;; `scroll-conservatively'). This is especially slow in larger files
      ;; during large-scale scrolling commands. If kept over 100, the window is
      ;; never automatically recentered.
      scroll-conservatively 101
      scroll-margin 0
      scroll-preserve-screen-position t
      ;; Reduce cursor lag by a tiny bit by not auto-adjusting `window-vscroll'
      ;; for tall lines.
      auto-window-vscroll nil
      ;; mouse
      mouse-wheel-scroll-amount '(2 ((shift) . hscroll))
      mouse-wheel-scroll-amount-horizontal 2)


    (setq vc-follow-symlinks t) ;; always follow symlinks
    (column-number-mode)
    (global-display-line-numbers-mode t) ;; display line numbers everywhere
    ;; (setq vc-follow-symlinks nil) ;; or never follow them

  (defun efs/display-startup-time ()
    (message "Emacs loaded in %s with %d garbage collections."
	     (format "%.2f seconds"
		     (float-time
		     (time-subtract after-init-time before-init-time)))
	     gcs-done))

  (add-hook 'emacs-startup-hook #'efs/display-startup-time)
\end{verbatim}

\subsection{Font settings}
\label{sec:org782d61d}
\begin{verbatim}
  ;; Font Configuration -----------------------
  ;; (set-face-attribute 'default nil :font "SauceCodePro Nerd Font 11")
  ;; IF FONT LOOKS WEIRD (TOO SLIM) then it means the font is not working properly, CHANGE IT

(cond ((eq system-type 'windows-nt)
    ;; Windows-specific code goes here.
    )
      ((eq system-type 'darwin)
      (set-face-attribute 'default nil :font "FiraCode Nerd Font" :height 170)

      ;; Set the fixed pitch face
      (set-face-attribute 'fixed-pitch nil :font "FiraCode Nerd Font" :height 180)

      ;; Set the variable pitch face
      (set-face-attribute 'variable-pitch nil :font "Cantarell" :height 180 :weight 'regular)
    )
    ((eq system-type 'gnu/linux)
      (set-face-attribute 'default nil :font "FuraCode Nerd Font" :height runemacs/default-font-size)

      ;; Set the fixed pitch face
      (set-face-attribute 'fixed-pitch nil :font "FuraCode Nerd Font" :height 120)

      ;; Set the variable pitch face
      (set-face-attribute 'variable-pitch nil :font "DejaVu Sans" :height 120 :weight 'regular)
    ))
  ;; -------------------------------------------------------
\end{verbatim}

\subsection{Spell-checking}
\label{sec:org6c3a70c}
\begin{verbatim}
;; execute spanish spell-checking on buffer
(defun flyspell-spanish ()
  (interactive)
  (ispell-change-dictionary "castellano")
  (flyspell-buffer))

(defun flyspell-english ()
  (interactive)
  (ispell-change-dictionary "default")
  (flyspell-buffer))
\end{verbatim}

\subsection{Package System}
\label{sec:org030bc8a}

\begin{verbatim}

;; Initialize package sources
(require 'package) ; bring in package module
; package repositories
(setq package-archives '(("melpa" . "https://melpa.org/packages/")
			 ("org" . "https://orgmode.org/elpa/")
			 ("elpa" . "https://elpa.gnu.org/packages/")))

(package-initialize) ; Initializes package system
(unless package-archive-contents ; unless package exists we refresh package list
 (package-refresh-contents)) 

;; Initialize use-package on non-Linux platforms
(unless (package-installed-p 'use-package) ; is this package installed, unless its installed install it
   (package-install 'use-package))
(require 'use-package)

(setq use-package-always-ensure t) ;; equivalent to writing :ensure t in all packages
;; makes sure that package is downloaded before use
\end{verbatim}

\subsection{General configurations}
\label{sec:orgf8c44c3}
\subsubsection{Tabs}
\label{sec:org320be3e}
\begin{verbatim}
;; (setq-default indent-tabs-mode nil)
;; (setq-default tab-width 4)
;; (setq indent-line-function 'insert-tab)
\end{verbatim}

\subsubsection{Log keystrokes on screen}
\label{sec:org3e8e5be}
\begin{verbatim}
;(use-package command-log-mode)
\end{verbatim}

\subsubsection{General configuration}
\label{sec:org72d0c0c}
\begin{verbatim}

(setq x-select-enable-clipboard-manager nil); weird emacs bug where it won't close

;; Make ESC quit prompts
(global-set-key (kbd "<escape>") 'keyboard-escape-quit)
(global-auto-revert-mode) ;;

\end{verbatim}

\subsubsection{Disable line numbers}
\label{sec:org403b8ac}
\begin{verbatim}
;; Disable line numbers for some modes
(dolist (mode '(org-mode-hook
		term-mode-hook
		eshell-mode-hook
		shell-mode-hook))
  (add-hook mode (lambda () (display-line-numbers-mode 0 ))))
\end{verbatim}
\subsection{Unused packages}
\label{sec:org40bd837}
\begin{verbatim}
;; (use-package langtool)

;; has to install pdf2svg on pc first
;; (use-package org-inline-pdf
;;   :init
;;   (add-hook 'org-mode-hook #'org-inline-pdf-mode))
\end{verbatim}
\subsection{Keep Folders Clean}
\label{sec:orgd8bcbbf}
\begin{verbatim}
;; NOTE: If you want to move everything out of the ~/.emacs.d folder
;; reliably, set `user-emacs-directory` before loading no-littering!
;(setq user-emacs-directory "~/.cache/emacs")

(use-package no-littering)

;; no-littering doesn't set this by default so we must place
;; auto save files in the same path as it uses for sessions
(setq auto-save-file-name-transforms
      `((".*" ,(no-littering-expand-var-file-name "auto-save/") t)))
\end{verbatim}
\subsection{UI settings}
\label{sec:org5827422}
\subsubsection{Ivy}
\label{sec:org5c70c55}
\begin{verbatim}
(use-package ivy ; makes navigation between stuff easier
  :diminish ; do not show stuff on bar or something
  :bind (("C-s" . swiper) ;;like / but with context
	 :map ivy-minibuffer-map
	 ("TAB" . ivy-alt-done)	
	 ("C-l" . ivy-alt-done)
	 ("C-j" . ivy-next-line)
	 ("C-k" . ivy-previous-line)
	 :map ivy-switch-buffer-map
	 ("C-k" . ivy-previous-line)
	 ("C-l" . ivy-done)
	 ("C-d" . ivy-switch-buffer-kill)
	 :map ivy-reverse-i-search-map
	 ("C-k" . ivy-previous-line)
	 ("C-d" . ivy-reverse-i-search-kill))
  :config
  (ivy-mode 1))
;; eval last sexp is better cause inconsistencies from hooks when running evalbuffer
;; and show keybindings
\end{verbatim}
\subsubsection{Ivy-rich better explanations}
\label{sec:org8ccf08a}
\begin{verbatim}
(use-package ivy-rich ;; shows better explanations
  :after ivy
  :init
  (ivy-rich-mode 1))
\end{verbatim}
\subsubsection{Improved Candidate Sorting with prescient.el}
\label{sec:orga12a20b}
prescient.el provides some helpful behavior for sorting Ivy completion candidates based on how recently or frequently you select them. This can be especially helpful when using M-x to run commands that you don’t have bound to a key but still need to access occasionally.

This Prescient configuration is optimized for use in System Crafters videos and streams, check out the video on prescient.el for more details on how to configure it!
\begin{verbatim}
(use-package ivy-prescient
  :after counsel ;; must have this
  ;; :custom
  ;; (ivy-prescient-enable-filtering nil) ;; keep ivy filtering style
  :config
  ;; Uncomment the following line to have sorting remembered across sessions!
  (prescient-persist-mode 1)
  (ivy-prescient-mode 1)
  )
(setq prescient-filter-method '(fuzzy regexp))
(setq prescient-sort-length-enable nil) ;; do not sort by length
\end{verbatim}

\begin{verbatim}
(use-package company-prescient
:after company
:config
(company-prescient-mode 1))

\end{verbatim}

\subsubsection{Counsel}
\label{sec:org60620d6}
\begin{verbatim}

;; With ivy-rich shows descriptions for commands 
(use-package counsel
:bind (("M-x" . counsel-M-x)
	("C-x b" . counsel-ibuffer)
	("C-x C-f" . counsel-find-file)
	:map minibuffer-local-map
	("C-r" . 'counsel-minibuffer-history))
	:config
	(setq ivy-initial-inputs-alist nil))
\end{verbatim}

\subsubsection{Doom}
\label{sec:orgaddaf39}
\begin{verbatim}
(use-package all-the-icons)
;; custom command line
(use-package doom-modeline
  :ensure t
  :init (doom-modeline-mode 1)
  :custom ((doom-modeline-height 15)))
(use-package doom-themes) ;; counsel-load-theme to load a theme from the list
(load-theme 'doom-one t) ;; if not using t will prompt if its safe to https://github.com/Malabarba/smart-mode-line/issues/100
\end{verbatim}
\subsection{Keybindings}
\label{sec:org5342b17}
\begin{verbatim}
(global-set-key (kbd "C-M-j") 'counsel-switch-buffer) ;; easier command to switch buffers
  ;; example (define-key emacs-lisp-mode-map (kbd "C-x M-t") 'counsel-load-theme) define keybinding only in emacs-lisp-mode

(use-package general ;; set personal bindings for leader key for example
 ; (general-define-key "C-M-j" 'counsel-switch-buffer) ;; allows to define multiple global keybindings
  ;; :after evil
  :config
  (general-create-definer rune/leader-keys
  :keymaps '(normal insert visual emacs)
  :prefix "SPC" 
  :global-prefix "C-SPC") ;; leader
  (rune/leader-keys ;; try to have similar keybindings in vim as well
   "t" '(:ignore t :which-key "toggles") ;; "folder" for toggles
   "b" '(:ignore b :which-key "buffers") 
   "h" '(:ignore h :which-key "git-gutter") 
   "c" '(org-capture :which-key "org-capture") ;; this is F*** awesome
   "g" '(git-gutter-mode :which-key "git-gutter toggle") 
   "hn" '(git-gutter:next-hunk :which-key "next hunk") 
   "hp" '(git-gutter:previous-hunk :which-key "previous hunk") 
   "hv" '(git-gutter:popup-hunk :which-key "preview hunk") 
   "hs" '(git-gutter:stage-hunk :which-key "stage hunk") 
   "hu" '(git-gutter:revert-hunk :which-key "undo hunk") ;; take back changes
   "hg" '(git-gutter :which-key "update changes") 
   "o" '(buffer-menu :which-key "buffer menu") 
   "bn" '(evil-next-buffer :which-key "next buffer") 
   "bp" '(evil-prev-buffer :which-key "previous buffer")
   "bc" '(evil-delete-buffer :which-key "close buffer")
   "bd" '(delete-file-and-buffer :which-key "delete file")
   "w" '(save-buffer :which-key "save buffer") ;; classic vim save
   "tt" '(counsel-load-theme :which-key "choose theme")))
\end{verbatim}

\subsubsection{Hydra}
\label{sec:orgfd40ec9}
\begin{verbatim}
(use-package hydra
  :defer t) ;; emacs bindings that stick around like mode for i3

(defhydra hydra-text-scale (:timeout 4)
  "scale text"
  ("j" text-scale-increase "in")
  ("k" text-scale-decrease "out")
  ("f" nil "finished" :exit t))
(rune/leader-keys
  "ts" '(hydra-text-scale/body :which-key "scale text"))

\end{verbatim}
\subsubsection{Evil}
\label{sec:org06e09af}
\begin{verbatim}
;; vim keybindings for easier on the fingers typing :D
(use-package evil
  :init
  (setq evil-want-integration t) ;; must have
  (setq evil-want-keybinding nil)
  (setq evil-want-C-u-scroll t)
  (setq evil-want-C-i-jump nil)
  :config
  (evil-mode 1)
  (define-key evil-insert-state-map (kbd "C-g") 'evil-normal-state)
  ;(define-key evil-insert-state-map (kbd "C-h") 'evil-delete-backward-char-and-join)

  ;; Use visual line motions even outside of visual-line-mode buffers
  (evil-global-set-key 'motion "j" 'evil-next-visual-line) ;; both of these
  (evil-global-set-key 'motion "k" 'evil-previous-visual-line) ;; are needed for org mode where g-j doesn't work properly

  (evil-set-initial-state 'messages-buffer-mode 'normal)
  (evil-set-initial-state 'dashboard-mode 'normal))
;; to center screen on cursor, zz or emacs-style C-l

;; https://github.com/linktohack/evil-commentary
;; use-package makes it so that it installs it from config and config section
;; activates the mode
(use-package evil-commentary
  :after evil
  :config
  (evil-commentary-mode))

(use-package evil-collection
  :after evil ;; load after evil, must have
  :config
  (evil-collection-init))

; C-z go back to EMACS MODE

\end{verbatim}

\subsection{Help}
\label{sec:org3206605}
\begin{verbatim}

(use-package which-key ;; This shows which commands are available for current keypresses
  :commands(helpful-callable helpfull-variable helpful-command helpful-key)
  :defer 0
  ;; runs before package is loaded automatically whether package is loaded or not we can also invoke the mode
  :diminish which-key-mode
  :config ;; this is run after the package is loaded
 (which-key-mode)
  (setq which-key-idle-delay 0.15)) ;; delay on keybindings 

(use-package helpful ;; better function descriptions
  :custom ;; custom variables
  (counsel-describe-function-function #'helpful-callable)
  (counsel-describe-variable-function #'helpful-variable)
  :bind
  ([remap describe-function] . counsel-describe-function) ;; remap keybinding to something different
  ([remap describe-command] . helpful-command) 
  ([remap describe-variable] . counsel-describe-variable))

\end{verbatim}

\subsection{Functions}
\label{sec:org26cace7}
\begin{verbatim}
(defun delete-file-and-buffer ()
  "Kill the current buffer and deletes the file it is visiting."
  (interactive)
  (let ((filename (buffer-file-name)))
    (if filename
	(if (y-or-n-p (concat "Do you really want to delete file " filename " ?"))
	    (progn
	      (delete-file filename)
	      (message "Deleted file %s." filename)
	      (kill-buffer)))
      (message "Not a file visiting buffer!"))))

\end{verbatim}

\subsection{Development}
\label{sec:orgbf06511}
\subsubsection{Rainbow-delimiters}
\label{sec:org6ebb379}
\begin{verbatim}
(use-package rainbow-delimiters
  :hook (prog-mode . rainbow-delimiters-mode)) ;; prog-mode is based mode for any programming language
\end{verbatim}

\subsubsection{IDE Features with lsp}
\label{sec:org78709e4}
M-? to find references, definition
\begin{verbatim}
(defun efs/lsp-mode-setup ()
  (setq lsp-headerline-breadcrumb-segments '(path-up-to-project file symbols))
  (lsp-headerline-breadcrumb-mode))

(use-package lsp-mode
  :commands (lsp lsp-deferred)
  :hook (lsp-mode . efs/lsp-mode-setup)
  :init
  (setq lsp-keymap-prefix "C-c l")  ;; Or 'C-l', 's-l'
  :config
  (lsp-enable-which-key-integration t)) ;; give description for keys with wichkey
\end{verbatim}
\subsubsection{lsp-ui}
\label{sec:org6a7b3e5}
\begin{verbatim}
(use-package lsp-ui
  :hook (lsp-mode . lsp-ui-mode)
  :custom
  (lsp-ui-sideline-show-code-actions t)
  (lsp-ui-doc-position 'bottom))
\end{verbatim}
\subsubsection{lsp-treemacs}
\label{sec:org0edba9b}
Tree views for different code aspects
\begin{verbatim}
(use-package lsp-treemacs
  :after lsp)
\end{verbatim}
\subsubsection{lsp-ivy}
\label{sec:orgf7fa193}
lsp-treemacs-symbols - Show a tree view of the symbols in the current file
lsp-treemacs-references - Show a tree view for the references of the symbol under the cursor
lsp-treemacs-error-list - Show a tree view for the diagnostic messages in the project

\begin{verbatim}
(use-package lsp-ivy
  :after (lsp-mode lsp))
\end{verbatim}
\subsubsection{Debugging with dap-mode}
\label{sec:org0d02a60}
\begin{verbatim}
(use-package dap-mode
  ;; Uncomment the config below if you want all UI panes to be hidden by default!
  ;; :custom
  ;; (lsp-enable-dap-auto-configure nil)
  ;; :config
  ;; (dap-ui-mode 1)
  :after lsp
  :config
  ;; Set up Node debugging
  (require 'dap-node)
  (dap-node-setup) ;; Automatically installs Node debug adapter if needed

  ;; Bind `C-c l d` to `dap-hydra` for easy access
  (general-define-key
    :keymaps 'lsp-mode-map
    :prefix lsp-keymap-prefix
    "d" '(dap-hydra t :wk "debugger")))
\end{verbatim}
\subsubsection{python}
\label{sec:org717331e}
Have to install
\begin{verbatim}
pip install python-lsp-server
\end{verbatim}


\begin{verbatim}
(use-package python-mode
  :ensure t
  :hook (python-mode . lsp-deferred)
  :custom
  ;; NOTE: Set these if Python 3 is called "python3" on your system!
  (python-shell-interpreter "python3")
  (dap-python-executable "python3")
  (dap-python-debugger 'debugpy)
  :config
  (require 'dap-python))
\end{verbatim}

\subsubsection{Latex}
\label{sec:orgeae761f}
install LSP server
\begin{verbatim}
cargo install --locked --git https://github.com/latex-lsp/texlab.git
\end{verbatim}

\begin{verbatim}
  (use-package latex-preview-pane
    :hook (latex-mode . latex-preview-pane-mode)
  )

;; (use-package tex
;;   :ensure auctex)
;;    (setq TeX-auto-save t)
;;   (setq TeX-parse-self t)
;;   (setq-default TeX-master nil)

;;   (add-hook 'LaTeX-mode-hook 'visual-line-mode)
;;   (add-hook 'LaTeX-mode-hook 'flyspell-mode)
;;   (add-hook 'LaTeX-mode-hook 'LaTeX-math-mode)

;;   (add-hook 'LaTeX-mode-hook 'turn-on-reftex)
;;   (setq reftex-plug-into-AUCTeX t)


\end{verbatim}

\begin{verbatim}
;; (use-package latex-mode
;;   :ensure t
;;   :hook (latex-mode . lsp-deferred)
;;   (add-hook 'latex-mode 'lsp-deferred)
;;   )
\end{verbatim}
\subsubsection{Company Mode}
\label{sec:org78673a7}
Company Mode provides a nicer in-buffer completion interface than completion-at-point which is more reminiscent of what you would expect from an IDE. We add a simple configuration to make the keybindings a little more useful (TAB now completes the selection and initiates completion at the current location if needed).
\begin{verbatim}
(use-package company
  :after lsp-mode
  :hook (lsp-mode . company-mode)
  :bind (:map company-active-map
	 ("<tab>" . company-complete-selection))
	(:map lsp-mode-map
	 ("<tab>" . company-indent-or-complete-common))
  :custom
  (company-minimum-prefix-length 1)
  (company-idle-delay 0.0))

(use-package company-box
  :hook (company-mode . company-box-mode))
\end{verbatim}
\subsubsection{Git}
\label{sec:org6f72740}
\begin{enumerate}
\item Magit
\label{sec:org3fb0b7b}
\begin{verbatim}
;; bring in the GIT
;; use C-x g to open magit status
;; type ? to know what can you do with magit
(use-package magit ;; use tab to open instead of za in vim
  :commands magit-status
  ;; :custom
  ;;   (magit-display-buffer-function #'magit-display-buffer-same-window-except-diff-v1)
  )

\end{verbatim}

\item Projects
\label{sec:org2eef22d}
\begin{verbatim}
;; emacs variables local to projects
(use-package projectile ;; git projects management
  :diminish projectile-mode
  :config (projectile-mode)
  :custom ((projectile-completion-system 'ivy)) ;; use ivy for completion can also use helm
  :bind-keymap
  ("C-c p" . projectile-command-map)
  :init
  ;; NOTE: Set this to the folder where you keep your Git repos!
  (when (file-directory-p "~/")
    (setq projectile-project-search-path '("~/")))
  (setq projectile-switch-project-action #'projectile-dired))

(use-package counsel-projectile ;; more commands with M-o in projectile (ivy allows that)
  :after projectile
  :config(counsel-projectile-mode)) 
\end{verbatim}
\item Gutter
\label{sec:org241f0f5}

\begin{verbatim}
(use-package git-gutter ;; works just like in vim :D
  :commands (git-gutter-mode git-gutter)
  :config
  ;; If you enable global minor mode
  ;; (global-git-gutter-mode t)
  ;; If you enable git-gutter-mode for some modes
  (add-hook 'ruby-mode-hook 'git-gutter-mode)
  )
\end{verbatim}

\item Unused packages
\label{sec:orge2096c5}

\begin{verbatim}
;; (use-package diff-hl
;;   :init
;;   (add-hook 'magit-pre-refresh-hook 'diff-hl-magit-pre-refresh)
;;   (add-hook 'magit-post-refresh-hook 'diff-hl-magit-post-refresh)
;;   :config
;;   (global-diff-hl-mode)
;;   (diff-hl-margin-mode)
;;   )
;; NOTE: Make sure to configure a GitHub token before using this package!
;; - https://magit.vc/manual/forge/Token-Creation.html#Token-Creation
;; - https://magit.vc/manual/ghub/Getting-Started.html#Getting-Started
;; (use-package forge) ;; more git functionality


\end{verbatim}
\end{enumerate}

\subsection{Org}
\label{sec:org07bb6b2}
\subsubsection{Templates}
\label{sec:orgd5656c2}
\begin{verbatim}
(with-eval-after-load 'org
    (require 'org-tempo)
    (add-to-list 'org-structure-template-alist '("py" . "src python"))
    (add-to-list 'org-structure-template-alist '("sh" . "src shell"))
    (add-to-list 'org-structure-template-alist '("hs" . "src haskell"))
    (add-to-list 'org-structure-template-alist '("cpp" . "src C++"))
    (add-to-list 'org-structure-template-alist '("el" . "src emacs-lisp"))
    )
\end{verbatim}

\subsubsection{Language support}
\label{sec:org5edf529}

\begin{verbatim}
(use-package haskell-mode
  :after org) ;; needed for haskell snippets

\end{verbatim}

\begin{verbatim}
(with-eval-after-load 'org
    (org-babel-do-load-languages
      'org-babel-load-languages
      '((emacs-lisp . t)
	(java . t)
	(python . t)))
    (push '("conf-unix" . conf-unix) org-src-lang-modes)
    )
\end{verbatim}

\subsubsection{Font setup}
\label{sec:org1a18b2f}
\begin{verbatim}
(defun efs/org-font-setup ()
  ;; Replace list hyphen with dot
  (font-lock-add-keywords 'org-mode
			  '(("^ *\\([-]\\) "
			     (0 (prog1 () (compose-region (match-beginning 1) (match-end 1) "•")))))) ;; replace - in lists for a dot

  ;; Set faces for heading levels
  (dolist (face '((org-level-1 . 1.2) ;; variable sizes for headers
		  (org-level-2 . 1.1)
		  (org-level-3 . 1.05)
		  (org-level-4 . 1.0)
		  (org-level-5 . 1.1)
		  (org-level-6 . 1.1)
		  (org-level-7 . 1.1)
		  (org-level-8 . 1.1)))
    (set-face-attribute (car face) nil :font "DejaVu Sans" :weight 'regular :height(cdr face)))

  ;; Ensure that anything that should be fixed-pitch in Org files appears that way
  (set-face-attribute 'org-block nil :foreground nil :inherit 'fixed-pitch)
  (set-face-attribute 'org-code nil   :inherit '(shadow fixed-pitch)) ;; fixed pitch on some stuff so that it lines up correctly, and variable on others so that it looks better
  (set-face-attribute 'org-table nil   :inherit '(shadow fixed-pitch))
  (set-face-attribute 'org-verbatim nil :inherit '(shadow fixed-pitch))
  (set-face-attribute 'org-special-keyword nil :inherit '(font-lock-comment-face fixed-pitch))
  (set-face-attribute 'org-meta-line nil :inherit '(font-lock-comment-face fixed-pitch))
  (set-face-attribute 'org-checkbox nil :inherit 'fixed-pitch))
\end{verbatim}

\subsubsection{Org configuration}
\label{sec:orgc470e09}
\begin{verbatim}
(defun efs/org-mode-setup ()
  (org-indent-mode)
  (variable-pitch-mode 1) ;; allows text to be of variable size
  (visual-line-mode 1) ;; makes emacs editing commands act on visual lines not logical ones, also word-wrapping, idk if i want this
  )

(use-package org  ;; org is already installed though
  :commands (org-capture org-agenda)
  :hook (org-mode . efs/org-mode-setup)
  :config
  (message "Org mode loaded")
  (setq org-ellipsis " ▾") ;; change ... to another symbol that is less confusing
  (efs/org-font-setup) ;; setup font
   ;; hides *bold* and __underlined__ and linked words [name][link]
  (setq org-agenda-start-with-log-mode t)
  (setq org-log-done 'time) ;; logs when a task goes to done C-h-v (describe variable)
  (setq org-log-into-drawer t) ;; collapse logs into a drawer
  (setq org-agenda-files
	'("~/fib/org/birthday.org"
	  "~/fib/org/Tasks.org"
	  "~/fib/org/Habits.org"
	  ))

  (require 'org-habit)
  (add-to-list 'org-modules 'org-habit) ;;  add org-habit to org-modules
  (setq org-habit-graph-column 60) ;; what column the habit tracker shows

  (setq org-todo-keywords
    '((sequence "TODO(t)" "NEXT(n)" "|" "DONE(d!)")
      (sequence "BACKLOG(b)" "PLAN(p)" "READY(r)" "ACTIVE(a)" "REVIEW(v)" "WAIT(w@/!)" "HOLD(h)" "|" "COMPLETED(c)" "CANC(k@)")))

  (setq org-refile-targets ;; move TODO tasks to a different file
    '(("Archive.org" :maxlevel . 1)
      ("Tasks.org" :maxlevel . 1)))

  ;; Save Org buffers after refiling!
  (advice-add 'org-refile :after 'org-save-all-org-buffers)

  (setq org-tag-alist
    '((:startgroup)
       ; Put mutually exclusive tags here
       (:endgroup)
       ("@errand" . ?E)
       ("@home" . ?H)
       ("@work" . ?W)
       ("agenda" . ?a)
       ("planning" . ?p)
       ("publish" . ?P)
       ("batch" . ?b)
       ("note" . ?n)
       ("idea" . ?i)))

;; Configure custom agenda views
  (setq org-agenda-custom-commands
   '(("d" "Dashboard"
     ((agenda "" ((org-deadline-warning-days 7)))
      (todo "NEXT"
	((org-agenda-overriding-header "Next Tasks")))
      (tags-todo "agenda/ACTIVE" ((org-agenda-overriding-header "Active Projects")))))

    ("n" "Next Tasks"
     ((todo "NEXT"
	((org-agenda-overriding-header "Next Tasks")))))

    ("W" "Work Tasks" tags-todo "+work-email")

    ;; Low-effort next actions
    ("e" tags-todo "+TODO=\"NEXT\"+Effort<15&+Effort>0"
     ((org-agenda-overriding-header "Low Effort Tasks")
      (org-agenda-max-todos 20)
      (org-agenda-files org-agenda-files)))

    ("w" "Workflow Status"
     ((todo "WAIT"
	    ((org-agenda-overriding-header "Waiting on External")
	     (org-agenda-files org-agenda-files)))
      (todo "REVIEW"
	    ((org-agenda-overriding-header "In Review")
	     (org-agenda-files org-agenda-files)))
      (todo "PLAN"
	    ((org-agenda-overriding-header "In Planning")
	     (org-agenda-todo-list-sublevels nil)
	     (org-agenda-files org-agenda-files)))
      (todo "BACKLOG"
	    ((org-agenda-overriding-header "Project Backlog")
	     (org-agenda-todo-list-sublevels nil)
	     (org-agenda-files org-agenda-files)))
      (todo "READY"
	    ((org-agenda-overriding-header "Ready for Work")
	     (org-agenda-files org-agenda-files)))
      (todo "ACTIVE"
	    ((org-agenda-overriding-header "Active Projects")
	     (org-agenda-files org-agenda-files)))
      (todo "COMPLETED"
	    ((org-agenda-overriding-header "Completed Projects")
	     (org-agenda-files org-agenda-files)))
      (todo "CANC"
	    ((org-agenda-overriding-header "Cancelled Projects")
	     (org-agenda-files org-agenda-files)))))))

 (setq org-capture-templates
    `(("t" "Tasks / Projects")
      ("tt" "Task" entry (file+olp "~/fib/org/Tasks.org" "Inbox")
	   "* TODO %?\n  %U\n  %a\n  %i" :empty-lines 1)

      ("j" "Journal Entries")
      ("jj" "Journal" entry
	   (file+olp+datetree "~/fib/org/Journal.org")
	   "\n* %<%I:%M %p> - Journal :journal:\n\n%?\n\n"
	   ;; ,(dw/read-file-as-string "~/Notes/Templates/Daily.org")
	   :clock-in :clock-resume
	   :empty-lines 1)
      ("jm" "Meeting" entry
	   (file+olp+datetree "~/fib/org/Journal.org")
	   "* %<%I:%M %p> - %a :meetings:\n\n%?\n\n"
	   :clock-in :clock-resume
	   :empty-lines 1)

      ("w" "Workflows")
      ("we" "Checking Email" entry (file+olp+datetree "~/fib/org/Journal.org")
	   "* Checking Email :email:\n\n%?" :clock-in :clock-resume :empty-lines 1)

      ("m" "Metrics Capture")
      ("mw" "Weight" table-line (file+headline "~/fib/org/Metrics.org" "Weight")
       "| %U | %^{Weight} | %^{Notes} |" :kill-buffer t)))

  )
\end{verbatim}

\subsubsection{Text in the middle}
\label{sec:org9b515f7}
\begin{verbatim}

;; (defun efs/org-mode-visual-fill ()
;;   (setq visual-fill-column-width 100 ;; set column width (character width?)
;;         visual-fill-column-center-text t) ;; center text on middle of screen
;;   (visual-fill-column-mode 1))

;; (use-package visual-fill-column
;;   :hook (org-mode . efs/org-mode-visual-fill))
\end{verbatim}
\subsubsection{Org bullets}
\label{sec:org452f434}
\begin{verbatim}
(use-package org-bullets ;; changes headers so that it doesn't show all of the stars
  :hook (org-mode . org-bullets-mode)
  :custom
  (org-bullets-bullet-list '("◉" "○" "●" "○" "●" "○" "●"))) ;; default symbols get weird
\end{verbatim}
\subsubsection{Org-fragtog}
\label{sec:org9ba0da0}
\begin{verbatim}
;; (use-package org-fragtog
;; :after org-bullets
;; :hook (org-fragtog) ; this auto-enables it when you enter an org-buffer, remove if you do not want this
;; :config
;; (org-fragtog-mode)
;; ;; whatever you want
;; )
\end{verbatim}
\subsubsection{Automatically tangle config file when we save it}
\label{sec:org00d9bf3}
\begin{verbatim}
;; Automatically tangle our Emacs.org config file when we save it
(defun efs/org-babel-tangle-config ()
  (when (string-equal (buffer-file-name)
		      (expand-file-name "~/dotfiles/dotfiles/.org/babel.org"))
    ;; Dynamic scoping to the rescue
    (let ((org-confirm-babel-evaluate nil))
      (org-babel-tangle))))
(add-hook 'org-mode-hook (lambda () (add-hook 'after-save-hook #'efs/org-babel-tangle-config))) ;; add hook to org mode
\end{verbatim}
\subsection{Dired}
\label{sec:orgdc8aa8e}

\begin{verbatim}
(use-package dired
  :ensure nil ;; make sure package manager doesn't try to install
  :commands (dired dired-jump)
  :bind (("C-x C-j" . dired-jump))
  :custom ((dired-listing-switches "-agho --group-directories-first"))
  :config
  (evil-collection-define-key 'normal 'dired-mode-map
    "h" 'dired-single-up-directory
    "l" 'dired-single-buffer))

(use-package dired-single
  :commands (dired dired-jump));; doesn't open new buffers like classic jump

(use-package all-the-icons-dired
  :hook (dired-mode . all-the-icons-dired-mode))

(use-package dired-open
  :commands (dired dired-jump)
  :config
  ;; Doesn't work as expected!
  ;;(add-to-list 'dired-open-functions #'dired-open-xdg t)
  (setq dired-open-extensions '(("png" . "feh") ;; use programs for file extensions
				("mkv" . "mpv"))))

(use-package dired-hide-dotfiles
  :hook (dired-mode . dired-hide-dotfiles-mode)
  :config
  (evil-collection-define-key 'normal 'dired-mode-map
    "H" 'dired-hide-dotfiles-mode))
\end{verbatim}

\subsection{Disable Optimization}
\label{sec:orgf89bee3}
Disable optimization at the end of startup so that garbage collector works properly and doesn't make emacs crash.
\begin{verbatim}
;; after startup, it is important you reset this to some reasonable default. A large 
;; gc-cons-threshold will cause freezing and stuttering during long-term 
;; interactive use. I find these are nice defaults:
  (setq gc-cons-threshold 16777216
	gc-cons-percentage 0.1
	file-name-handler-alist last-file-name-handler-alist)
\end{verbatim}
\end{document}